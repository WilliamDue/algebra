\documentclass{article}
\usepackage{graphicx}
\usepackage{amsthm}
\usepackage{amsfonts}
\usepackage{amsmath}
\usepackage{float}
\usepackage{amssymb}
\usepackage{enumitem}
\usepackage{xcolor}

\title{Section 1.1}

\begin{document}

\maketitle
\section{Exercise}
\begin{enumerate}[label=(\alph*)]
    \item Is not associative since: $(a - a) - a = -a$ and $a - (a - a) = a$
    i.e. $a \neq -a$.
    \item Is associative since:
    \begin{align*}
        a \star (b \star c) &= a + (b + c + bc) + a(b + c + bc) \\
        &= a + b + c + bc + ab + ac + abc \\
        &= a + b + ab + c + ac + bc + abc \\
        &= (a + b + ab) + c + (a + b + ab)c \\
        &= (a \star b) \star c 
    \end{align*}
    \item Is not associative since:
    \begin{align*}
        a \star (b \star c) &= \frac{a + \frac{b + c}{5}}{5} \\
        &= \frac{\frac{5a}{5} + \frac{b + c}{5}}{5} \\
        &= \frac{5a + b + c}{10} \\
        &\neq \frac{a + b + 5c}{10} \\
        &= \frac{\frac{a + b}{5} + \frac{5c}{5}}{5} \\
        &= \frac{\frac{a + b}{5} + c}{5} \\
        &= (a \star b) \star c 
    \end{align*}
    \item Is not associative since:
    \begin{align*}
        (a, b) \star ((c, d) \star (e, f)) &= (a, b) \star (cf + de, df)  \\
        &= (a(cf + de) + bdf, bdf) \\
        &= (acf + ade + bdf, bdf) \\
        &= (acf + ade + bdf, bdf) \\
        &\neq (adf + bcf + bde, bdf) \\
        &= ((ad + bc)f + bde, bdf) \\
        &= (ad + bc, bd) \star (e, f) \\
        &= ((a, b) \star (c, d)) \star (e, f)
    \end{align*}
    \item Is not associative since:
    \begin{align*}
        a \star (b \star c) &= \frac{a}{\frac{b}{c}} \\
        &= a \left(\frac{b}{c}\right)^{-1} \\
        &= a \left(\frac{c}{b}\right) \\
        &= \frac{ac}{b} \\
        &\neq \frac{a}{bc} \\
        &= \left(\frac{a}{b}\right) c^{-1} \\
        &= \frac{\frac{a}{b}}{c} \\
        &= a \star (b \star c)
    \end{align*}
\end{enumerate}
\section{Exercise}
\begin{enumerate}[label=(\alph*)]
    \item Is not commutative since $a - 2a = -a \neq a = 2a - a$.
    \item Is commutative since:
    \begin{align*}
        a \star b &= a + b + ab \\
        &= b + a + ba \\
        &= b \star a
    \end{align*}
    \item Is commutative since:
    \begin{align*}
        a \star b &= \frac{a + b}{5} \\
        &= \frac{b + a}{5} \\
        &= b \star a
    \end{align*}
    \item Is commutative since:
    \begin{align*}
        (a, b) \star (c, d) &= (ad + bc, bd) \\
        &= (cb + ad, db) \\
        &= (c, d) \star (a, b) 
    \end{align*}
    \item Is not commutative since $\frac{1}{2} \neq \frac{2}{1}$
\end{enumerate}
\section{Exercise}
\begin{proof}
    Let $\overline{a}, \overline{b}, \overline{c} \in \mathbb{Z}/n\mathbb{Z}$.
    Using $\overline{a} + \overline{b} = \overline{a + b}$ we can show that:
    \begin{align*}
        (\overline{a} + \overline{b}) + \overline{c} &= \overline{a + b} + \overline{c} \\
        &= \overline{a + b + c} \\
        &= \overline{a} + \overline{b + c} \\
        &= \overline{a} + (\overline{b} + \overline{c}) \\
    \end{align*}
\end{proof}
\section{Exercise}
\begin{proof}
    Let $\overline{a}, \overline{b}, \overline{c} \in \mathbb{Z}/n\mathbb{Z}$.
    Using $\overline{a} \cdot \overline{b} = \overline{a \cdot b}$ we can show that:
    \begin{align*}
        (\overline{a} \cdot \overline{b}) \cdot \overline{c} &= \overline{a \cdot b} \cdot \overline{c} \\
        &= \overline{a \cdot b \cdot c} \\
        &= \overline{a} \cdot \overline{b \cdot c} \\
        &= \overline{a} \cdot (\overline{b} \cdot \overline{c}) \\
    \end{align*}
\end{proof}
\section{Exercise}
\begin{proof}
    We wish to show that $(\mathbb{Z}/n\mathbb{Z}, \cdot)$ for any $n > 1$ is
    not a group. For $(\mathbb{Z}/n\mathbb{Z}, \cdot)$ to be a group then there
    must exists an inverse $\overline{a}$ such that $\overline{a} \cdot
    \overline{0} = \overline{1}$. If we take the representative of
    $\overline{a}$ to be $a$ and the representative of $\overline{0}$ to be $0$
    then we get $a \cdot 0 = 0$ for any $a$ so $\overline{a} \cdot \overline{0}
    = \overline{0}$. Hence no inverse exists of $\overline{0}$ and therefore
    $(\mathbb{Z}/n\mathbb{Z}, \cdot)$ is not a group.
\end{proof}
\section{Exercise}
\begin{enumerate}[label=(\alph*)]
    \item It is easy to see that the inverse $\frac{-a}{2b + 1}$ and identity
    $0$ exisist. Addition is also associative so remaining concern is if $+$
    always results in an element with an uneven denominator.
    \begin{align*}
        \frac{a}{2b + 1} + \frac{c}{2d + 1} &= \frac{a(2d + 1) +  c(2b + 1)}{(2b + 1)(2d + 1)} \\
        &= \frac{a(2d + 1) +  c(2b + 1)}{4bd + 2b + 2d + 1} \\
        &= \frac{a(2d + 1) +  c(2b + 1)}{2(2bd + b + d) + 1} \\ 
    \end{align*}
    The denominator is always uneven so the set is closed under addition.
    \item Once again we only need to show the denominator is even.
    \begin{align*}
        \frac{a}{2b} + \frac{c}{2d} &= \frac{a2d +  c2b}{2(b2d)} 
    \end{align*}
    The denominator is always even so the set is closed under addition.
    \item This is not a valid group since $\frac{1}{2}$ is an element in the set
    but $\frac{1}{2} + \frac{1}{2} = \frac{1}{1}$ is too large to be in set,
    hence the set is not closed under addition.
    \item This is not a valid group since both $2$ and $\frac{-2}{3}$ are
    members but $2 + \frac{-2}{3} = \frac{1}{2}$ is too small to be an element
    in the set, hence the set is not closed under addition.
    \item Clearly an inverse always exists $\frac{-a}{1}$ or $\frac{-a}{2}$ and
    we have the identity $0$ and an associative operation. It remains to show
    that the set is closed under addition. 
    \begin{align*}
        \frac{a}{1} + \frac{b}{1} &= \frac{a + b}{1} \\
        \frac{a}{2} + \frac{b}{2} &= \frac{a + b}{2} \\
        \frac{a}{2} + \frac{b}{1} &= \frac{a + 2b}{2}
    \end{align*}
    As one can see it is closed under addition.
    \item First we observe that $\frac{1}{2} + \frac{1}{3} = \frac{5}{6}$, since
    there does no exist a number which both divide $5$ and $6$ then the fraction
    can not be simplied and we have an element which can does not belong to the
    original set. Meaning this set is not closed under addition and therefore
    not an group.
\end{enumerate}
\section{Exercise}
\begin{proof}
    By definition we have that $G$ must be closed under $\star$ since we always
    remove the integral part leading to $0 \leq x \star y < 1$.

    Now we wish to show that the operation is associative, first we realize:
    \begin{align*}
        [x - [y]] = [x] - [y] 
    \end{align*}
    This holds since $[y]$ is already an integer so we can subtract it later.
    Now we can derive the following and show it is associative:
    \begin{align*}
        (x \star y) \star z &= (x \star y) + z - [x \star y + z] \\
        &= (x + y - [x + y]) + z - [(x + y - [x + y]) + z] \\
        &= x + y + z - [x + y + z - [x + y]] - [x + y] \\
        &= x + y + z - ([x + y + z] - [x + y]) - [x + y] \\
        &= x + y + z - [x + y + z] + [x + y] - [x + y] \\
        &= x + y + z - [x + y + z] \\
        &= x + y + z - [x + y + z] + [y + z] - [y + z] \\
        &= x + y + z - ([x + y + z] - [y + z]) - [y + z] \\
        &= x + y + z - [x + y + z - [y + z]] - [y + z] \\
        &= x + (y + z - [y + z]) - [x + (y + z - [y + z])] \\
        &= x \star (y + z - [y + z]) \\
        &= x \star (y \star z) 
    \end{align*}
    We have an identity $0$ since:
    \begin{align*}
        x \star 0 = x + 0 - [x + 0] = x - [x] = 0 + x - [0 + x] = 0 \star x
    \end{align*}
    And we have an inverse $-x$ since:
    \begin{align*}
        x \star -x &= x + (-x) - [x + (-x)] \\
        &= 0 - [0] \\
        &= 0 \\
        &= 0 - [0] \\
        &= (-x) + x - [(-x) + x] \\
        &= -x \star x
    \end{align*}
    Hence $(G, \star)$ must be a group.
\end{proof}
\section{Exercise}
\begin{enumerate}[label=(\alph*)]
    \item
    \begin{proof}
        First it must be shown that $(G, \cdot)$ is closed under
        multiplciation. Given $z, w \in G$ then it must hold for some $n, m \in
        \mathbb{Z}^+$ that $(z^n)^x \cdot (w^m)^y = 1$ for all $x, y \in
        \mathbb{R}$. Since $z^n = w^m = 1$ and $1^x = 1$ for all $x \in \mathbb{R}$
        we can derive:
        \begin{align*}
            1 &= \left(z^n\right)^m \cdot \left(w^m\right)^n \\
            &= z^{mn} \cdot w^{mn} \\
            &= \left(zw\right)^{nm} 
        \end{align*}
        Hence $G$ is closed under multiplciation since $\left(zw\right)^{nm} = 1$,
        therefore $zw \in G$. Lastly $(G, \cdot)$ is a group since $1 \in G$ is the
        identity. The inverse to $z$ is $z^{n - 1}$ since $z \cdot z^{n - 1} = z^n =
        1$ for some $n \in \mathbb{Z}^+$ and $z^{n - 1} \in G$ because $(z^{n -
        1})^n = z^{n \cdot n - n} = \frac{z^{n \cdot n}}{z^n} = z^{n \cdot n} =
        (z^{n})^n = 1$. And multiplication is associative.
    \end{proof}

    \item
    \begin{proof}
        $(G, +)$ is not a group since $1 \in G$ but $1 + 1 = 2 \notin G$
        since there exists no $n \in \mathbb{Z}^+$ such that $2^n = 1$ hence $G$ is
        not closed under addition and therefore not a group.
    \end{proof}
\end{enumerate}
\section{Exercise}
\begin{enumerate}[label=(\alph*)]
    \item
    \begin{proof}
        $(G, +)$ is closed under addition since given $a + b\sqrt{2},
        c + d\sqrt{2} \in G$ then:
        \begin{align*}
            a + b\sqrt{2} + c + d\sqrt{2} = (a + c) + (b + d)\sqrt{2} \in G
        \end{align*}
        $(G, +)$ is also a group since since $0$ is the identity, the inverse to
        $a + b\sqrt{2}$ is $(-a) + (-b\sqrt{2})$, and addition is associative.
    \end{proof}
    \item
    \begin{proof}
        $(G, \cdot)$ is closed under addition since given $a + b\sqrt{2},
        c + d\sqrt{2} \in G$ then:
        \begin{align*}
            (a + b\sqrt{2}) \cdot (c + d\sqrt{2}) &= c(a + b\sqrt{2}) + d\sqrt{2}(a + b\sqrt{2}) \\
            &= ca + cb\sqrt{2} + ad\sqrt{2} + db(\sqrt{2})^2 \\
            &= (ca + db2) + (cb + ad)\sqrt{2} \in G
        \end{align*}
        $(G, \cdot)$ is also a group since since $1$ is the identity, the inverse to
        $a + b\sqrt{2}$ is $\frac{1}{a + b\sqrt{2}}$ which is in $G$ since:
        \begin{align*}
            \frac{1}{a + b\sqrt{2}} &= \frac{a - b\sqrt{2}}{(a - b\sqrt{2})(a + b\sqrt{2})} \\
            &= \frac{a - b\sqrt{2}}{a(a - b\sqrt{2}) + b\sqrt{2}(a - b\sqrt{2})} \\
            &= \frac{a - b\sqrt{2}}{a^2 - ab\sqrt{2} + ab\sqrt{2} - b^2(\sqrt{2})^2} \\
            &= \frac{a - b\sqrt{2}}{a^2 - 2b^2} \\
            &= \frac{a}{a^2 - 2b^2} - \frac{b}{a^2 - 2b^2}\sqrt{2} \in G \\
        \end{align*}
        And addition is associative.
        \end{proof}
\end{enumerate}
\section{Exercise}
\begin{proof}
    Let $(G, +)$ be an abelian group then for elements $g_i, g_j \in G$ where $i, j
    \in \mathbb{Z}^+$ are the indices in the group table $M$. Since $(G, +)$ is an
    abelian group then:
    \begin{align*}
        g_i + g_j = g_j + g_i
    \end{align*}
    So it must hold that $M_{ij} = M_{ji}$ henece $M$ is symmetric.
    
    Let the group table $M$ be an symmetric matrix i.e. $M_{ij} = M_{ji}$ then it
    must hold that $g_i + g_j = g_j + g_i$, hence $(G, +)$ is an abelian group.
\end{proof}
\section{Exercise}
\begin{align*}
    \overline{0}^1 = 0; |\overline{0}| = 1 \\
    \overline{1}^12 = 0; |\overline{1}| = 12 \\
    \overline{2}^6 = 0; |\overline{2}| = 6 \\
    \overline{3}^4 = 0; |\overline{3}| = 4 \\
    \overline{4}^3 = 0; |\overline{4}| = 3 \\
    \overline{5}^{12} = 0; |\overline{5}| = 12 \\
    \overline{6}^2 = 0; |\overline{6}| = 2 \\
    \overline{7}^{12} = 0; |\overline{7}| = 12 \\
    \overline{8}^{3} = 0; |\overline{8}| = 3 \\
    \overline{9}^{4} = 0; |\overline{9}| = 4 \\
    \overline{10}^{6} = 0; |\overline{10}| = 6 \\
    \overline{11}^{12} = 0; |\overline{11}| = 12 
\end{align*}
\section{Exercise}
\begin{align*}
    \overline{1}^1 = 1; |\overline{1}| = 1 \\
    \overline{-1}^{2} = \overline{11}^{2} = 1; |\overline{-1}| = 2 \\
    \overline{5}^{2} = 1; |\overline{5}| = 2 \\
    \overline{-7}^{2} = \overline{5}^{2} = 1; |\overline{-7}| = 2 \\
    \overline{13}^{1} = \overline{1}^{1} = 1; |\overline{13}| = 2
\end{align*}
\section{Exercise}
\begin{align*}
    \overline{1}^{36} = 0; |\overline{1}| = 36 \\
    \overline{2}^{18} = 0; |\overline{2}| = 18 \\
    \overline{6}^{6} = 0; |\overline{6}| = 6 \\
    \overline{9}^{4} = 0; |\overline{9}| = 4 \\
    \overline{10}^{18} = 0; |\overline{10}| = 18 \\
    \overline{12}^{3} = 0; |\overline{12}| = 3 \\
    \overline{-1}^{36} = 0; |\overline{-1}| = 36 \\
    \overline{-10}^{18} = 0; |\overline{-10}| = 18 \\
    \overline{-18}^{2} = 0; |\overline{-18}| = 2
\end{align*}
\section{Exercise}
\begin{align*}
    \overline{1}^{1} = 1; |\overline{1}| = 1 \\
    \overline{-1}^{2} = 1; |\overline{-1}| = 2 \\
    \overline{5}^{6} = 1; |\overline{5}| = 6 \\
    \overline{13}^{3} = 1; |\overline{13}| = 3 \\
    \overline{-13}^{6} = 1; |\overline{-13}| = 6\\
    \overline{17}^{2} = 1; |\overline{17}| = 2
\end{align*}
\section{Exercise}
\begin{proof}
    Let $P(n): (a_1a_2\cdots a_n)^{-1} = a_{n}^{-1}a_{n - 1}^{-1} \cdots
    a_{1}^{-1}$, we wish to show this predicate holds for $2 \leq n$ by
    induction. By Proposition 1. we have the base case $P(2): (a_1a_2)^{-1} =
    a_{2}^{-1}a_{1}^{-1}$. Now assume $P(n)$ holds, we can then show taht $P(n +
    1)$ holds by Proposition 1:
    \begin{gather*}
        (a_1a_2\cdots a_n)^{-1} = a_{n}^{-1}a_{n - 1}^{-1} \cdots a_{1}^{-1} \\
        \iff \\
        a_{n + 1}^{-1} (a_1a_2\cdots a_n)^{-1} = a_{n + 1}^{-1}a_{n}^{-1}a_{n - 1}^{-1} \cdots a_{1}^{-1} \\
        \iff \\
        (a_1a_2\cdots a_n a_{n + 1})^{-1} = a_{n + 1}^{-1}a_{n}^{-1}a_{n - 1}^{-1} \cdots a_{1}^{-1}
    \end{gather*}
\end{proof}
\section{Exercise}
\begin{proof}
    First we wish to show given $x^2 = 1$ then $|x| = 1$ or $|x| = 2$ exists, by
    $x^2 = 1$ we have: 
    \begin{gather*}
        x^2 = 1 \iff x = x^{-1} 
    \end{gather*}
    So $x$ is its own inverse and either $x = 1$ or $x \neq 1$. The first case
    exists for any group $1 \cdot 1 = 1$ and we have $|x| = 1$. The second case
    exists since there exists a group $(\mathbb{Z}/4\mathbb{Z}, +)$ where
    $\overline{2} + \overline{2} = \overline{0}$ so we have $|x| = 2$.
    
    Let $x^2 = 1$ then it must hold that $|x| \leq 2$ and if $x^n = 1$ for $n
    > 2$ then $|x| \leq 2$ since $1$ or $2$ is smaller than $n$.
    
    Let $|x| \leq 2$ then $x^n = 1$ for $1 \leq n \leq 2$, if $x^1 = 1$ then $x
    = 1$ and $x^2 = 1$ holds, so $x^2 = 1$ holds for any $|x| \leq 2$. So the
    biimplication holds in both directions.
\end{proof}
\section{Exercise}
\begin{proof}
    Let $|x| = n$ for some $n > 1$ then:
    \begin{align*}
        x^n = 1 &\iff x^nx^{-1} = x^{-1} \\
        & \iff x^{n - 1}xx^{-1} = x^{-1} \\
        &\iff x^{n - 1} = x^{-1}
    \end{align*}
\end{proof}
\section{Exercise}
\begin{proof}
    Let $xy = yx$ then:
    \begin{align*}
        xy = yx &\iff y^{-1} xy = y^{-1}y x \\
        &\iff y^{-1} xy = x \\
        &\iff x^{-1}y^{-1} xy = x^{-1}x \\
        &\iff x^{-1}y^{-1} xy = 1
    \end{align*}
\end{proof}
\section{Exercise}
\begin{enumerate}[label=(\alph*)]
    \item Note: I am not totally satified by this proof but it seems like you
    would have to do something like this if you want to use the definition of
    $x^n$
    
    \begin{proof}
        Let $a, b \in \mathbb{Z}^+$ by definition of $x^ax^b$ then we have:
        \begin{align*}
            x^ax^b = \underbrace{xx \cdots x}_{a\text{ times}}\underbrace{xx \cdots x}_{b\text{ times}}
        \end{align*}
        If we have $a$ times $x$ and $b$ times $x$ multiplied together then we
        must in total have $a + b$ times $x$:
        \begin{align*}
            \underbrace{xx \cdots x}_{a + b\text{ times}} = x^{a + b}
        \end{align*}
        Now by definition of $(x^a)^b$ we have:
        \begin{align*}
            (x^a)^b = \underbrace{x^ax^a \cdots x^a}_{b \text{ times}} = \underbrace{\underbrace{xx\cdots x}_{a \text{ times}} \underbrace{xx\cdots x}_{a \text{ times}} \cdots \underbrace{xx\cdots x}_{a \text{ times}}}_{b \text{ times}}
        \end{align*}
        Hence we have $b$ times of $a$ times $x$ i.e. $ab$ of $x$:
        \begin{align*}
            \underbrace{\underbrace{xx\cdots x}_{a \text{ times}} \underbrace{xx\cdots x}_{a \text{ times}} \cdots \underbrace{xx\cdots x}_{a \text{ times}}}_{b \text{ times}} = x^{ab}
        \end{align*} 
    \end{proof}
\end{enumerate}
\end{document}